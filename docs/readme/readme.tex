\documentclass{article}
\usepackage{hyperref}
\newcommand{\DIP}{DIP}
\newcommand{\FPGA}{FPGA}
\newcommand{\FPGAs}{\FPGA s}
\newcommand{\ghc}{\textit{ghc}}
\newcommand{\GUI}{GUI}
\newcommand{\LED}{LED}
\newcommand{\LEDs}{\LED s}
\newcommand{\IDE}{IDE}
\newcommand{\ISE}{ISE}
\newcommand{\LLFSM}{LLFSM}
\newcommand{\LLFSMs}{\LLFSM s}
\newcommand{\MiPal}{MiPal}
\newcommand{\VHDL}{VHDL}

\newcommand{\exampleBlock}[1]{\begin{quote}\texttt{#1}\end{quote}}
\newcommand{\eitem}[1]{\item[\$] #1}
\begin{document}
    \title{Spartan \LLFSM}
    \date{\today}
    \author{Morgan McColl}
    \maketitle
    \tableofcontents
    \newpage
    \section{Introduction}
        The purpose of this paper is to give some documentation of the \textit{Spartan \LLFSM} project. I have originally designed this project to allow users to implement \LLFSMs\/ on the \textit{minispartan6+} evaluation board. The board utilises a \textit{Xilinx Spartan-6 XL25} \FPGA.
    \section{Setup}
        Before you can start building machines in \VHDL, you will need to install a number of dependencies. The following steps are for an ubuntu machine.
        \subsection{\ISE}
            \ISE\/ is the \IDE\/ used for the \textit{Spartan-6} \FPGA.
            \begin{enumerate}
                \item Create a Xilinx account at \href{url}{https://www.xilinx.com/}.
                \item Download and install the Xilinx \ISE\/ webpack found at \href{URL}{\url{https://www.xilinx.com/support/download/index.html/content/xilinx/en/downloadNav/design-tools/v2012_4---14_7.html}}.
                \item Create a license by navigating to \href{URL}{\url{https://www.xilinx.com/member/forms/license-form.html}}
                \item Download and place the license file in \texttt{~/.Xilinx}. Optionally you can run the Xilinx license manager which is installed with \ISE.
            \end{enumerate}
            \subsection{xc3sprog}
                \textit{xc3sprog} is used to transfer the bit file from your computer to the board.
                \exampleBlock{
                    \eitem svn checkout https://svn.code.sf.net/p/xc3sprog/code/trunk xc3sprog-code
                    \eitem cd xc3sprog-code
                    \eitem cmake .
                    \eitem make
                    \eitem make install
                }
                You might also require additional libraries:
                \begin{itemize}
                    \item \textit{libftdi}
                    \exampleBlock{
                        \eitem{sudo apt-get install libftdi1}
                    }
                    \item \textit{libftd2xx} located at \href{URL}{\url{http://www.ftdichip.com/Drivers/D2XX.htm}}.
                \end{itemize}
            \subsection{miniSProg (optional)}
                This program provides a \GUI\/ for xc3sprog. It requires qt4/5 and \textit{qtcreator} to install.
                \exampleBlock{
                    \eitem{sudo apt-get install build-essential}
                    \eitem{sudo apt-get install qtcreator}
                    \eitem{sudo apt-get install qt5-default}
                    \eitem{git clone https://github.com/scarabhardware/miniSProg}
                    \item[] Open the project using \textit{qtcreator} (the project file is located at \texttt{./miniSProg/miniSProg/miniSProg.pro}).
                    \item[] Build the project.
                }
            \subsection{Haskell}
                You also need \ghc.
                \exampleBlock{
                    \eitem{sudo apt-get install haskell-platform}
                }
            \subsection{Additional \MiPal\/ Software}
                You will need an \IDE\/ to create the machine format (MiCase, MiPalCase, MiEdit (not tested)).
            \subsection{Installing Spartan \LLFSM}
                \begin{enumerate}
                    \item Navigate to the interpretor folder in the root directory of this repository.
                    \exampleBlock{
                        \eitem{make}
                        \eitem{make install}
                    }
                    \item This command will create a binary called spartanllfsm in \texttt{/usr/local/bin}
                \end{enumerate}
\end{document}