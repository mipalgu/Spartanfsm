\documentclass{article}
\usepackage{minted}
\begin{document}
	\title{Pixel Flip Report}
	\author{Morgan McColl}
	\date{\today}
	\maketitle
	\section{Introduction}
		The aim of this report is to summarise the Pixel Flip project. The Pixel Flip project is used to show that LLFSMs can be run in parallel to achieve a result. The machines are responsible for flipping a pixel from red to green and green to red. The pixels are then sent out through a hdmi port to a tv where it can be visually displayed.
	\section{The LLFSM}
		The machine has 3 external variables: The clock (input), redOut (output) and greenOut(output) and 2 internal variables: pastRed and pastGreen. The machine had 4 states: red, green, compare and wait. The red and green states set the redOut/greenOut variables to x'FF while setting the greenOut/redOut variables to x'00 respectively. The onExit of these states set the past internal variables to their respective values. They then transition to a wait state which simply increments a variable till it reaches 50million (the clock frequency) i.e. enough clock ticks to last 1 second. The wait state then transitions to the compare state. The compare state checks the pastVariables and transitions accordingly: if the pastRed variable is set then it transitions to the green state, if the pastGreen variable is set then it transitions to the red state. The cycle then continues indefinitely.
	\section{Generating An Image}
		An array was used to store the output from the machines. These were declared in the architecture sections which can be seen below:
		
		\begin{minted}{vhdl}
constant h_rez        : natural := 1280;
constant h_sync_start : natural := 1280+72;
constant h_sync_end   : natural := 1280+80;
constant h_max        : natural := 1647;
signal   h_count      : unsigned(11 downto 0) := (others => '0');

constant v_rez        : natural :=720;
constant v_sync_start : natural := 720+3;
constant v_sync_end   : natural := 720+3+5;
constant v_max        : natural := 720+29;
signal   v_count      : unsigned(11 downto 0) := (others => '0');
constant width: integer := 50;
constant height: integer := 50;
type column is array ((height - 1) downto 0) of std_logic_vector(7 downto 0);
type screen is array ((width - 1) downto 0) of column;

signal greenOut: screen;
signal redOut: screen;
		\end{minted}
			
		The component is also declared:
		
		\begin{minted}{vhdl}
component PixelFlip
 Port(
     clk50: in std_logic;
     redOut: out std_logic_vector(7 downto 0);
     greenOut: out std_logic_vector(7 downto 0)
 );
end component;
		\end{minted}
		
		The LLFSMs are generated for a height and width:
		
		\begin{minted}{vhdl}
PixelY: for I in 0 to (width - 1) generate
    PixelX: for J in 0 to (height - 1) generate
        PixelFlipGen: PixelFlip port map(
            clk50 => clk75,
            redOut => redOut(i)(j),
            greenOut => greenOut(i)(j)
        );			
    end generate PixelX;
end generate PixelY;
		\end{minted}
		
		In a process statement, an if-statement is used to decide whether to take output from the array or out a blue pixel:
		
		\begin{minted}{vhdl}
if h_count < width and v_count < height then
    red <= redOut(to_integer(h_count))(to_integer(v_count));
    green <= greenOut(to_integer(h_count))(to_integer(v_count));
    blue <= (others => '0');
    blank <= '0';
else
    red <= (others => '0');
    green <= (others => '0');
    blue <= (others => '1');
    blank <= '0';
end if;
		\end{minted}
	\section{Problems with the Optimiser}
		Basically since the encoders take a single input at every clock cycle, the LLFSMs are getting optimised away. To get around this, I disabled an option in the compilation settings and the LLFSMs are correctly being created. I tried using a draw area of 50x50 but this took up the entire space on the FPGA. I have got it working with a 2x2 space but I am confident that it can do at least a 10x10. It does legitimately run these LLFSMs in parallel with this flag disabled so this project has achieved its aim.
	\section{Code}
		Top Module:
		\inputminted{vhdl}{../../projects/PixelFlip/dvid_serdes.vhd}
		clocking:
		\inputminted{vhdl}{../../projects/PixelFlip/clocking.vhd}
		dvid\_out:
		\inputminted{vhdl}{../../projects/PixelFlip/dvid_out.vhd}
		tmds\_encoder:
		\inputminted{vhdl}{../../projects/PixelFlip/tmds_encoder.vhd}
		output\_serialiser:
		\inputminted{vhdl}{../../projects/PixelFlip/output_serialiser.vhd}
		Pixel Flip LLFSM:
		\inputminted{vhdl}{../../projects/PixelFlip/PixelFlip.vhd}		
		Image generation:
		\inputminted{vhdl}{../../projects/PixelFlip/vga_gen.vhd}		
\end{document}